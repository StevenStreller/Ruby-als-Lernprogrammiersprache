\documentclass[12pt,DIV=14, version=first, BCOR=10mm,a4paper,twoside,parskip=half-,headsepline,headinclude]{scrartcl}
% Grundgröße 12pt, zweiseitig
% Standardpakete
\usepackage[utf8]{inputenc}
\usepackage[T1]{fontenc}
\usepackage{lmodern}
% deutsche Silbentrennung
\usepackage[ngerman]{babel}
% schöne Tabellen
\usepackage{booktabs}
% etwas Mathematik
\usepackage{amsmath,amssymb}
% Grafiken einbinden
\usepackage{graphicx}
% Kopf- und Fußzeilen
\usepackage[headsepline,footsepline,automark]{scrlayer-scrpage}
\pagestyle{scrheadings}
% Links anklickbar
\usepackage{url,hyperref,color} 
\definecolor{MidnightBlue}{cmyk}{0.98,0.13,0,0.43}
\hypersetup{colorlinks,breaklinks=true, pdffitwindow=true,pdfpagelayout=SinglePage,
            linkcolor=blue,pdfauthor={}, pdfdisplaydoctitle=true,
            pdfsubject={},   % to be defined later
            bookmarksnumbered,citecolor=MidnightBlue, 
            urlcolor=MidnightBlue}
\raggedbottom
\renewcommand{\topfraction}{1}
\renewcommand{\bottomfraction}{1}


% zum Ausprobieren
\usepackage{blindtext}


% jetzt gehts los
\begin{document}% hier gehts los
  \thispagestyle{empty} % Titelseite
\includegraphics[width=0.2\textwidth]{Wortmarke_WI_schwarz}

   {  ~ \sffamily
  \vfill
  {\Huge\bfseries Ruby als Lernprogrammiersprache}
  \bigskip

  {\Large 
  Nicolai Böttger und Jon-Steven Streller \\[2ex]
 Seminar-Arbeit im Studiengang "`Angewandte Informatik"'
 \\[5ex]
   \today } 
}
 \vfill
  
  ~ \hfill
  \includegraphics[height=0.3\paperheight]{H_WI_Pantone1665} 

\vspace*{-3cm}

  \newpage \thispagestyle{empty}
 \begin{tabular}{ll}
{\bfseries\sffamily Autor 1:} &  Nicolai Böttger \\ 
            & 1476431 \\
            & \href{mailto:nicolai.boettger@stud.hs-hannover.de}{nicolai.boettger@stud.hs-hannover.de} \\
            & Verfasste Seiten/Abschnitte: ...
            \\[5ex]
{\bfseries\sffamily Autor 2:} & Jon-Steven Streller \\ 
            & 1475759 \\
            & \href{mailto:steven.streller@stud.hs-hannover.de}{steven.streller@stud.hs-hannover.de} \\
           & Verfasste Seiten/Abschnitte: ... \\[5ex]
 {\bfseries\sffamily Prüfer:} &Prof. Dr. Dennis Allerkamp \\
          & Abteilung Informatik, Fakultät IV \\
         & Hochschule Hannover \\
        & \href{mailto:dennis.allerkamp@hs-hannover.de}{dennis.allerkamp@hs-hannover.de}
\end{tabular}

\vfill

\begin{center} \sffamily\bfseries Selbständigkeitserklärung \end{center}
% fett und zentriert in der minipage

Mit der Abgabe der Ausarbeitung erklären wir, dass wir die eingereichte Seminar-Arbeit
selbständig und ohne fremde Hilfe verfasst, andere als die von uns angegebenen Quellen
und Hilfsmittel nicht benutzt und die den benutzten Werken wörtlich oder
inhaltlich entnommenen Stellen als solche kenntlich gemacht haben. 
\vspace*{7ex}

Hannover, den \today \hfill 

\pagebreak

  \pdfbookmark[0]{Inhalt}{contents}
  \tableofcontents  % Inhaltsverzeichnis

\pagebreak

%\input{abkuerz.tex}      % Einbinden von Tex-Files
%\input{einfuehrung.tex}


% durch eigenen Text ersetzen

\section{Einführung}
\begin{flushleft}
Die vorliegende Arbeit entstand im Rahmen des Seminars \textit{\glqq Programmiersprachen für den Einstieg\grqq}. Ziel des Seminars war es mehrere Programmiersprachen auf ihre Tauglichkeit im Anwendungsbereich einer pädagogischen Lehrveranstaltung zu analysieren. Hierbei wurde jede Programmiersprache einzeln von verschiedenen Studierenden in Zweiergruppen vorgestellt und die jeweiligen Ergebnisse Ihrer Forschung präsentiert. Nachfolgend entstand ein Austausch zwischen den Studierenden über positive und auch negative Auffälligkeiten in der Präsentation.
\end{flushleft}

\subsection{Vorgehensweise}
\begin{flushleft}
Um eine sinnvolle Bewertung der Sprache in Bezug zum Thema des Seminars zu erreichen haben wir zuerst nach Studien gesucht, welche Kriterien definieren, nach denen man eine Programmiersprache bewerten kann. Danach haben wir angefangen uns selbst in unsere vorzustellende Sprache \textit{\glqq Ruby\grqq} einzuarbeiten. Nachdem wir ein grundsätzliches Verständnis der Sprache bekommen haben, haben wir angefangen \textit{\glqq Ruby\grqq} in Bezug zu den vorher gewählten Kriterien zu bewerten.
\end{flushleft}

\subsection{Aufbau der Arbeit}
\begin{flushleft}
Im folgenden \textbf{Hauptteil} wird die Programmiersprache \textit{\glqq Ruby\grqq} vorgestellt. Hierbei werden auf besondere Merkmale der Sprache eingegangen und \textit{Anwendungsbeispiele} visualisiert. Anschließend werden die Methodik und die Kriterien erläutert und definiert. Zum Abschluss werden die Forschungsergebnisse in Form einer Bewertung dargelegt.
\end{flushleft}

\section{Hauptteil}

\subsection{Was ist Ruby?}
\begin{flushleft}
Im Jahr 1995 veröffentlichte \textit{Yukihiro Matsumoto} die erste Version (Version 0.95) von \textit{\glqq Ruby\grqq}. \textit{Yukihiro Matsumoto} lies sich während der Entwicklung von \textit{\glqq Ruby\grqq} von mehreren Programmiersprachen wie z.B. Perl, Smalltalk, Lisp, Ada inspirieren. \textit{Ruby} besitzt eine tief integrierte 
Objektorientierung\footnote{\label{foot:1} Ein System besteht in der Objektorientierung ausschließlich aus Objekten, die miteinander über Nachrichten kommunizieren. Jedes Objekt verfügt über Eigenschaften und Methoden. Die Eigenschaften beschreiben dabei über ihre Werte den Zustand eines Objektes, die Methoden die möglichen Handlungen eines Objektes. vgl. [WL01]} (kurz OO).
\end{flushleft}

\subsection{Arbeitsweise von Ruby}
\begin{flushleft}
An dieser Stelle wird kurz auf die interne Arbeitsweise von Ruby eingegangen. Ruby-Code wird ein Statement nach dem anderen, von einem Interpreter übersetzt und ausgeführt. Im Gegensatz zum Compiler, analysiert dieser den Code schneller und das Schrittweise übersetzen erleichtert das Finden von Fehlern im Code, allerdings hat ein kompiliertes Programm meist eine bessere Performance.
\end{flushleft}

\subsection{Methodik}
\begin{flushleft}
Um die Programmiersprache Ruby in Bezug zur Fragestellung, ob sie sich als Programmiersprache für den Einstieg eignet bewerten zu können, wurde als Methodik ein Kriterienkatalog erstellt, welcher die Sprache in hinsichtlich verschiedener Bereiche untersucht. Dieser Katalog besteht aus den Kriterien Einstiegsfreundlichkeit, Skalierbarkeit, Verständlichkeit, Dokumentation, Verbreitung der Sprache und Ausstattung der Schule. Eine besondere Gewichtung wurde hierbei auf die Kriterien Skalierbarkeit und Verständlichkeit gelegt, da diese die grundlegenden Aspekte zum Verstehen der Sprache sind. Nach einer kurzen Erläuterung der Kriterien, werden diese in Bezug zur Programmiersprache Ruby gesetzt und es wird erörtert, inwiefern das Kriterium zutrifft (trifft voll zu, trifft teilweise zu, trifft nicht zu)
\end{flushleft}

\subsubsection{Einstiegsfreundlichkeit}
\begin{flushleft}
Das Kriterium der \textit{Einstiegsfreundlichkeit} fast alles zusammen, was benötigt wird um mit dem Programmieren in der Sprache anzufangen. So sollte die Installation von möglichen Compilern oder Interpretern nicht zu aufwändig sein und ohne große Fachkentnisse vorgenommen werden können. Auch wäre es von Vorteil, wenn die Installationsbestandteile nicht allzu viel Speicher einnehmen. Ebenfalls wird hier berücksichtigt, ob beispielsweise mehrere Dateien angelegt, oder andere Vorgänge wie ein mögliches Linken von Dateien geschehen muss, ehe man ein geschriebenes Programm ausführen kann.

Die technischen Vorraussetzungen, die benötigt werden, um mit der Programmiersprache \textit{\glqq Ruby\grqq} arbeiten zu können beschränken sich ausschließlich auf einen \textit{Ruby-Interpreter}, welchen man auf der offiziellen Website von \textit{\glqq Ruby\grqq} herunterladen kann. Auch die anschließende Installation läuft unter dem Betriebssystem \textit{Microsoft Windows} so ab, wie man es von anderen heruntergeladenen Programmen gewohnt ist. Die Größe der Standardedition dieses Interpreters beläuft sich auf ca. 60MB und nimmt somit vergleichsweise wenig Speicher auf der Festplatte ein (Java SE Development Kit 12.0.1 ~160MB).
Theoretisch könnte man in \textit{\glqq Ruby\grqq} ein komplettes Programm in nur einer Datei schreiben, welches man ohne weitere Zwischenschritte von einem Terminal aufrufen könnte. Zusätzlich wird mit dem Download des Interpreters das Programm \textit{IRB} (Interactive Ruby) installiert, mit welchem man unmittelbares Feedback auf eingegebenen Ruby-Code erhält.

Dementsprechend trifft das gewählte Kriterium der \textit{Einstiegsfreundlichkeit} voll zu.
\end{flushleft}

\subsubsection{Skalierbarkeit}
\begin{flushleft}
Unter Skalierbarkeit wird alles bewertet, was mit den spezifischen Eigenschaften und Besonderheiten der Sprache zu tun hat. Wie hoch ist die Komplexität um eine Ausgabe zu erzeugen? Wie sehr bindet mich die Programmiersprache an bestimmte Syntax und Semantik? Ein weiterer Punkt der nicht zu vernachlässigen ist, ist die Plattformunabhängigkeit der Sprache. Hierzu wird im nachfolgenden Abschnitt näher drauf eingegangen.
\end{flushleft}

\subsubsection{Verständlichkeit}
\begin{flushleft}
Bei der \textit{Verständlichkeit} geht es vor allem, dass der geschriebene Quellcode möglichst simpel und verständlich wirkt. Ein abstrakter Quellcode wäre nicht fördernd um grundlegende Konzepte des Programmieren zu vermitteln. Außerdem ist ein schnelles und klares Feedback des Interpreters essentiell.

Die Programmiersprache \textit{\glqq Ruby\grqq} besitzt eine sehr hohe Skalierbarkeit. Auch wenn \textit{\glqq Ruby\grqq} eine tief integrierte Objektorientierung (nachstehend \glqq (OO)\grqq) \, besitzt, ist es nicht zwingend erforderlich das Konzept der eigentlichen Objektorientierung zu verstehen. In \textit{\glqq Ruby\grqq} ist es auch nicht nötig einen expliziten Einstiegspunkt zu definieren, wie es zum Beispiel in Java mit der klassischen Main-Methode passiert. Was das erlernen der Programmiersprache \textit{\glqq Ruby\grqq} auch deutlich vereinfacht, ist dass \textit{\glqq Ruby\grqq} an der englischen Sprache sehr angelehnt ist und einer gesprochenen Sprache angelehnt ist. Dies äußert sich öfter in Quellcodebeispielen, bei denen man einzelne Abschnitte wie gesprochene Sätze ablesen kann. Der \textit{Interpreter} hilft aktiv bei der Fehlersuche im Quellcode und somit ist der Grad der Frustration bei nichtfunktionierenden Code geringer als bei Programmiersprachen die auf einen \textit{Kompilieren} (engl. Compiler) setzen. 

Dementsprechend trifft das gewählte Kriterium der \textit{Verständlichkeit} voll zu.
\end{flushleft}

\subsubsection{Dokumentation}
\begin{flushleft}
Eine aktuelle und feinsäuberlich ausgearbeitete \textit{Dokumentation} der Programmiersprache ist überlebenswichtig für jene. Beachtet werden muss auch zum Beispiel wie viel Lehr/-Sachbücher über die Programmiersprache publiziert wurden. Ergänzend hierzu ist es immer sehr positiv, wenn die Programmiersprache über viele Entwickler/-innen verfügt, da so ein großer Support bei technischen Fragen eher gegeben ist. Damit Schüler auch von Zuhause aus programmieren können, wäre es hilfreich, wenn spezifische Webseiten mit Tutorials oder Lernvideos zum Erlernen dieser Programmiersprache existieren würden.

Hinsichtlich Literatur, also Lehr- und Sachbüchern lässt die Dokumentation der Sprache \textit{\glqq Ruby\grqq} zu wünschen übrig. Es gibt zwar einige Bücher, diese sind aber in der Regel schon etwas älter und dementsprechend nicht mehr kompatibel mit dem heutigen Stand der Sprache. Besonders schwierig ist es gute Literaturbücher in der deutschen Sprache zu finden (https://wiki.ruby-portal.de/Literatur.html). Im Gegensatz dazu gibt es aber einige gute Videotutorials, entweder kostenlos auf Youtube oder kostenpflichtig - teilweise mit Übungsaufgaben - auf Plattformen wie {\glqq Udemy\grqq} um sich das Programmieren mit \textit{\glqq Ruby\grqq} selbst beizubringen.
Trotz der Tatsache, dass man die Grundlagen der Programmiersprache so auch gut über das Internet lernen kann, trifft das Kriterium einer ausgereiften \textit{Dokumentation}, aufgrund fehlender schriftlicher Literatur nur teilweise zu.
\end{flushleft}

\subsubsection{Verbreitung der Sprache}
\begin{flushleft}
Unbeliebte Programmiersprachen sind in der Regel auch nicht so stark in der Arbeitswelt gefordert, weshalb es sehr ernüchternd sein kann, eine Programmiersprache pädagogisch vermittelt zu bekommen die keine reelle Perspektive in der Zukunft besitzt. Zusätzlich ist eine Würdigung beziehungsweise eine Akzeptanz in der \glqq Programmiergesellschaft\grqq \, wünschenswert, da es durchaus vorkommt, dass Programmiersprachen sich eher schwer durchsetzen, wenn diese nicht von der o.g Zielgruppe akzeptiert werden. Desweiteren ist eine aktive Weiterentwicklung der Programmiersprache eine elementare Voraussetzung um gegen andere aktive und etablierte Sprachen zu konkurrieren.
\end{flushleft}

\subsubsection{Ausstattung der Schule}
\begin{flushleft}
Selbstverständlich sollten die Lehrmittel beziehungsweise Lizenzkosten im Optimalfall keine Kosten verursachen, um auch Schulen mit wenig Budget eine Perspektive bieten zu können. Ideal wäre ein/e Lehrbeauftragter/-in die bereits grundlegende Kenntnisse in der Programmiersprache beherrscht. Somit könnten Ausbildungsmaßnahmen für den/die Lehrbeauftragte/-n eingespart werden und die eingesparten Kosten in die Lehrveranstaltung investiert werden
\end{flushleft}


\section{Schlussbemerkungen}
\begin{flushleft} 

\end{flushleft}

\pagebreak

\section{Literaturverzeichnis}
\begin{flushleft}
\href{https://wirtschaftslexikon.gabler.de/definition/objektorientierung-43907} {[WL01] Wirtschaftslexikon Gabler}
\end{flushleft}

% Literatur
\bibliographystyle{alpha}
%\bibliography{mybib}

\end{document}

